% $Id: spices.tex,v 1.1 1997/06/25 19:30:54 dmg Exp $

\documentstyle[10pt,fullpage,isolatin1,html,array]{article}
\title{Flavouring the World. The FAQ about Spices}

\author{Daniel M. Germ\'{a}n\\Dept. of Computer Science\\University of
  Waterloo\\ Waterloo Ont. N2L 3G1\\ 
  dmg@csg.uwaterloo.ca\\http://csgwww.uwaterloo.ca/\~{ }dmg/}

\begin{document}
\renewcommand{\topfraction}{0.95}
\renewcommand{\textfraction}{0.02}

\bibliographystyle{ieeetr}


{\Large Frequently Asked Questions about Spices Ver. 1.1 (Jun 25, 1997.)}

Last additions:

\begin{itemize}
\item Patricia Rain address and email updated.
\end{itemize}



%% Version 0.25 Dic. 9, 1995
%%\begin{abstract}
%%  Frequently Ask Questions about Spices (Draft Ver. 0.25) 
%%  \begin{itemize}
%%  \item A table of contents has been included.
%%  \item Scannings of Vanilla, pepper, whitepepper, cinnamon were added.
%%  \item The hypertext version has been divided by sections
%%  \end{itemize}
%%\end{abstract}
%%
%% Version 0.2 Dic. 5, 1995
%%%\begin{titlepage}
%%%  \thispagestyle{empty}
%%  \maketitle
%%  \begin{abstract}
%%    Frequently Ask Questions about Spices (Draft Ver. 0.2) 
%%    \begin{itemize}
%%    \item Major rewriting. Many typos, misspellings, and wrongly written paragraphs were
%%      amended.
%%    \item A new section on coconut milk.
%%    \item Some answers regarding vanilla were rephrased to avoid
%%      confusions.
%%    \item Added some entries in uses of spices.
%%    \end{itemize}
%%  \end{abstract}
%%%\end{titlepage}
%%
%%% Version 0.1, 28 Nov. 95
\tableofcontents


\section{Introduction}
This FAQ describes basic facts about spices: their nature, storage,
and use.

%%This is a FAQ draft and it will be posted weekly to
%%rec.food.cooking until it becomes an official USENET FAQ.

This FAQ is posted montly to the following newsgroups:
rec.food.cooking, rec.food.veg, rec.food.preserving, rec.answers, and
news.answers.

This FAQ is (C) Copyright 1995 Daniel M. Germ�n. This text, in whole
or in part, may not be sold in any medium, including, but not limited
to electronic, CD-ROM, or published in print, without the explicit,
written permission of Daniel M. Germ�n. This FAQ can be reproduced and
distributed electronically or in hardcopy as long as this is done for
free and it is kept intact.

If you have any comments about this document, please direct them to
{\sl dmg@csg.uwaterloo.ca}.

The hypertext version of this FAQ is available at:

\begin{quote}
\htmladdnormallink{http://csgwww.uwaterloo.cahttp://csgwww.uwaterloo.ca/~dmg/faqs/spices/}
  {http://csgwww.uwaterloo.ca/\~{ }dmg/faqs/spices/}
\end{quote}

\section{Spices}


\subsection{What are spices}

Spices are the various strongly flavoured or aromatic substances of
vegetable origin, commonly used as condiments or employed for other
purposes on account of their fragance and preservation qualities
\cite{oed}.

\subsection{Why are spices so tasty?}
Spices have two main components \cite{Pruthi}:

\begin{itemize}
\item Volatile oils. Also known as essential oils, they are
  responsible for the characteristic aroma of spices. 
\item Oleoresins, or non volatile extracts, which are responsible for
  the typical taste and flavour.
\end{itemize}

\subsection{What is the difference between essential oils and oleoresins?}

By David Soknacki, from Econ Manufacturing:

\begin{quote}
  \emph{ ``Essential oils are generally produced by injecting the
    spice bed with steam, and then separating the distillate into the
    essential oil and water. On the other hand oleoresins are produced
    by soaking spices in a solvent, whether a combination of ethanol
    and water in your example for vanilla, or hexane in the case of
    many of our spices. One of the final stages in processing is to
    remove the solvent to acceptable levels (35%
    ethanol for vanilla, but under 25ppm for hexane in spices). What
    is left are all of the flavour components dissolved by the
    solvent.  Companies decide between essential oils and oleoresins
    usually depending on the flavour profile they require for their
    finished product.''  }
\end{quote}


\subsection{Names of Spices}

The following table summarizes the common and scientific names of
most popular spices and the part of the plant they come from.

%\begin{table*}[htbp]
%  \begin{center}
%    \leavevmode
%\footnotesize
\begin{tabular}{|l| >{\em} l|l|}
\hline
\textbf{Common Name} & \textbf{Scientific Name} & \textbf{Part of the plant}\\
\hline\hline
Allspice &  Pimenta dioca & Berries \\
Anise & Pimpinella aisum & Seed \\
Annato & Bixa orellana &  Seeds \\
Basil & Ocimum basilicum &  Leaves \\
Bay &  Laurus nobilis & Leaves \\
Caraway & Carum carvi & Seeds \\
Cardamon & Elettaria cardamomum & Seeds \\
Celery & Apium graveolens &  Seeds \\
Cayenne & Capsicum annuum &  Podlike berries \\
Chia & Salvia columbariae & Seeds \\
Chile Pepper & Capsiums & Berries \\
Cassia & Cinnamomum cassia & Bark \\
Chives & Allium schoenoprasum & Leaves \\
Chocolate & Theobroma cacao &  Seeds \\
Cinnamon & Cinnamomum zeylanicum & Bark \\
cloves & Syzygium aromaticum & Flower buds \\
Coffee & Coffea arabica & Seeds \\
Coriander & Coriandrum sativum & Seeds \\
Cumin &  Cuminum cyminum &  Seeds \\
Dill &  Anethum graveolens & Leaves seeds \\
Fennel &  Foeniculum vulgare & Seeds \\
Fenugreek & Trigonella foenumgraecum \\
Garlic & Allium sativum & Bulb \\
Ginger & Zaingiber offinale & Rhizomes \\
Horseradish & Armoracia rusticana & Roots \\
Mace & Myristica fragrans & Seed coverings (arils) \\
Marjoram & Sweet marjoram & Leaves \\
Mint & Mentha species & Seeds \\
Nutmeg & Myristica fragrans & Peeled seeds \\
Onion & Allium cepa & Bulbs \\
Oregano & Origanum vulgare & Leaves \\
Paprika & Capsicum annuum & Fruit pods \\
Parsley & Petroselinum crispum & Leaves \\
Pepper & Piper nigrum & Buds \\
Pimiento & Capsicum annuum & Fruits \\
Poppy seed & Papaver somniferum & Seeds \\
Rosemary & Rosmarinus officialis & Leaves or flowers \\
Safflower & Carthamus tinctorius & Flowers \\
Saffron & crocus sativus & Flowers stigmas \\
Sage & salvia species & Leaves \\
Savory & Satureja species & Leaves \\
Sesame & Sesamum indicum & Seeds \\
Shallot & Allium cepa & Bulbs \\
Star Anise & Illicium verum & Unripe fruits \\
Tarragon & Artemisia dracunculus & Leaves \\
Thyme & Thymus species & Leaves \\
Turmeric & Curcuma domestica & Rhizomes \\
Vainilla & Vanilla planifolia & Seed pods \\
\hline
\end{tabular}
%    \caption{Common and Scientific Name of Spices}
%    \label{tab:names}
%  \end{center}
%\end{table*}


\subsection{What are some uses of spices (excluding the kitchen)}

Some examples on the use of spices:

\begin{itemize}
\item Antioxygenic properties. Some spices retard the oxydation of
  fat.
\item Preserving action. 
  Some spices contain essential oils that are toxic to microorganisms \cite{Pruthi}:
  \begin{itemize}
  \item Cloves contain plenty of essential oil (15 to 20\%); its main
    component --eugenol, 80 to 92\%-- inhibits the growth of
    microorganisms. 
  \item At normal growth temperatures, the mustard's essential oil is
    toxic to microorganism.
  \end{itemize}
\item Antimicrobial activity. 
   Black pepper, garlic,  cinnamon, nutmeg,
  cloves, ginger, cumin, and caraway amongst others, are used in
  India for correcting and a variety of intestinal disorders
  \cite{Pruthi}. \\
  In a study, Subrahmanyan, et.al \cite{Subra} reported the
  susceptibilities of \textsl{E. coli} to garlic: at a concentration
  of 20 mg/ml of garlic, the number of organisms per ml. were 17,
  22, and 300 after 0, 6, and 24 hrs. respectively; for the same
  periods, at a concentration of 0 mg/ml, the results were: 17, 3600,
  and 16800.
%\item Physilogical and medical effects.
%  \begin{itemize}
%  \item Stimulation of salivary flow and amylasis (the promotion of
%    the growth of amylase --any of the enzymes (as amylopsin) that
%    accelerate the hydrolysis of starch and glycogen or their
%    intermediate hydrolysis products). Some spices, when added to food,
%    promote increased, salivation compared to unspiced food. Citric
%    acid, acetic acid, and tartaric acid are the best promoters,
%    followed by the biting spices --capsicums (chili), ginger,
%    pepper \cite{Pruthi}.
%%  \item Stimulation of gastric secretion.  It is debatable whether
%    pations with peptic ulcer should avoid spices. Several studies
%    show that it promotes gastric secretion
%  \end{itemize}
\item Perfumery and cosmetics. 
  \begin{itemize}
  \item Oils from cardamon, cumin, celery, chive, juniper, and nutmeg
    are used in different types of perfume \cite{Pruthi}.
  \item The oil of cinnamon, dill seed, fennel seed, and nutmeg are
    used in scenting soaps, dental preparations, hair lotions, and
    others \cite{Pruthi}.
  \end{itemize}
\end{itemize}

\section{Pepper}

\begin{rawhtml}
  <img src="http://csgwww.uwaterloo.ca/~dmg/faqs/spices/gifs/pepper_16.gif"><p>
\end{rawhtml}


\subsection{What is black pepper?}

Black pepper is the whole dried immature fruit of the {\em Piper
  nigrum}.

\subsection{Where is pepper native from?}

It is native of the Western Ghats in India, where it is still
restricted as a wild plant. Nowadays, it can also be found growing
wild in north Burma and the hills of Assam.

\subsection{Where the name pepper comes from}

It is believed that the name comes from the Sanskrit {\em pippali},
which was the name of the long pepper, {\em P. longum}, which is
now never seen in Europe.


\subsection{What is green pepper?}

It is unripe, but fully developed, pepper which is artificially dried
or preserved in ``wet'' form, e.g. brine, vinegar, citric acid.

\subsection{What is white pepper?}

\begin{rawhtml}
  <img src="http://csgwww.uwaterloo.ca/~dmg/faqs/spices/gifs/whipepc_16.gif"><p>
\end{rawhtml}

According to Pruthi \cite{Pruthi}, there are several methods to prepare it:

\begin{enumerate}
\item Water steeping and rotting technique
  \begin{itemize}
  \item From ripening fresh berries. It is the oldest method. Fresh
    berries are harvested when one or two berries start turning yellow
    or red. There are submerged for several days, at the eleventh the
    skin is removed by hand or mechanical methods. The berries
    --without skin-- are washed and immerse in a bleaching solution.
    After 2 days, then they are washed and dried.
  \item From dried berries. Pepper berries are dried for 7 to 10 days,
    then submerged for one or two weeks. Again they are washed,
    bleached, washed and dried.
  \end{itemize}
\item Steaming. Ripening green berries are steamed for 10 to 15
  minutes, then a machine removes the skin. Also, the berries are
  treated with a bleaching solution, then washed and dried.
%\item Chemical technique. Joshi patented the following method:
%  \begin{enumerate}
%  \item dried black berries are immerse in five times its weight of
%    water for 4 days
%  \item treated with 4\% NaOH solution and boiling the mixture
%  \item removing the skins by stirring at 1600 r.p.m.
%  \item bleaching with H\_2O\_2
%  \item drying the berries at 125 F.
%  \end{enumerate}
\item Decortication technique (also known as \textsl{decorticated
    pepper}) . Created by decortication machines that remove the skin
  of the dried black peppercorns. 
\end{enumerate}


\subsection{What is pink pepper?}

Pink pepper is the berry from the Schinus terebinthifolius, a South
American tree. They are midly toxic. \cite{Mulherin}

%%No.  Green peppercorns are indeed real, underripe peppercorns, piper
%%nigrum, which are preserved as Daniel describes above.  And the
%%pink "peppercorns" that are used in "gourmet" cooking are not
%%true peppercorns, BUT the ones that are used in cooking are perfectly
%%safe, FDA-approved, and are dried berries from the Baies rose
%%plant.  The pink berries of the "ornamental pepper tree" can
%%apparently cause severe allergic reactions, as you said; but
%%these are NOT the pink peppercorns that one finds in food stores
%%or "gourmet peppercorn blends".


%-- 
%Dan Masi
%Mentor Graphics Corp.
%dan_masi@mentorg.com

\subsection{Are there any differences between white and black pepper?}


The only significant difference between white and black pepper is in
starch and fiber content. {\em The belief that white pepper is milder
  in flavour than black pepper does not seem to be confirmed by the
  scientific data\/} \cite{Pruthi}. However, there are some
differences in pungency --of black and white pepper-- due to
geographical origin.

\subsection{Storage}

Pepper can be washed and re-dried before grinding. Store away from
sunlight at moderate temperatures and low humidity. Only ground
pepper needs to be stored in sealed containers.

Pepper loses more volatile oils the finer it is ground.


\section{Cinnamon}

\begin{rawhtml}
  <img src="http://csgwww.uwaterloo.ca/~dmg/faqs/spices/gifs/cinna_16.gif"><p>
  <img src="http://csgwww.uwaterloo.ca/~dmg/faqs/spices/gifs/canela_16.gif"><p>
\end{rawhtml}



Dried bark of {\em Cinnamomum verum} (syn. {\em C. zeylanicm}).

\subsection{Where does Cinnamon come from?}
It is indigenous in Sri Lanka, which still produces the largest
quantity and best quality. Seychelles is the second largest producer.

\section{Vanilla}
\begin{rawhtml}
  <img src="http://csgwww.uwaterloo.ca/~dmg/faqs/spices/gifs/vanilla_16.gif"><p>
\end{rawhtml}

Vanilla is the fully grown fruit of the orchid {\em Vanilla
  fragrans\/} %and {\em Vanilla planifolia\/}
harvested before it is fully ripe; then it is fermented and cured. The
fruits are usually referred to as vanilla beans \cite{Purseglove-2}.
Vanilla production is regulated by ISO standard 5565.

\subsection{Where does Vanilla come from?}
Vanilla is native to Mexico, Guatemala and other parts of Central
America. At the present time, it grows also in Madagascar, the
Seychelles, Tahiti, R�union and other tropical areas\cite{Mulherin}.
The first recorded use of the spice in European literature dates back
to 1520, when Moctezuma II offered vanilla flavoured chocolate to
Hern�n Cort�s. However, the use of {\em tlilxochitl\/} (Nahuatl for
vanilla) is earlier documented in the precolumbian literature.

\subsection{What is Vanillin?}
Vanillin is a crystalline phenolic aldehyde C\_8H\_8O\_3 that is the
chief fragrant component of vanilla and is used especially in flavouring and
in perfumery \cite{Webster}.

Vanillin can now be produced synthetically, and it is much cheaper
than natural vanilla.

\subsection{Products}

\subsubsection{What is Vanilla Extract?}
\label{vanilla-extract}
Vanilla extract is obtained by macerating the cured beans in a
solution of water and alcohol. It might contain sugar or glycerine as
sweeteners or thickeners \cite{Purseglove-2}.

Conventional vanilla extracts have a minimum ethanol content of 35\%,
and contain the soluble extractives from 1 part by weight of vanilla
beans in 10 parts by volume of hydroalcoholic solution.
\cite{Purseglove-2}.

\subsubsection{How do I differentiate between real and unreal vanilla
  extract?}

``The two best indicators of pure vanilla extract are alcohol content
and price.  The alcohol content must be at least 35\%; synthetics
usually have no alcohol or at most, about 2\%. Any purchases that cost
less than US\$25.00 a quart are most likely
synthetic.''\cite{patricia-rain}


\subsubsection{What is vanilla flavouring?}
\label{vanilla-flavouring}
It is similar to vanilla extract (see \ref{vanilla-extract}) but
contains less than 35\% of ethanol per volume.

\subsubsection{What is vanilla tincture?}
It is used exclusively in pharmaceutical applications. It is prepared
by maceration from 1 part of vanilla beans by weight to 10 parts of
hydroalcoholic solution and contains added sugar. It differs from
vanilla extract (see \ref{vanilla-extract}) by having at least a 38\%
ethanol content.

\subsubsection{What is concentrated vanilla extract and concentrated vanilla
  flavouring?} 
They are prepared by removing the solvent from their
regular counterparts (see \ref{vanilla-extract},
\ref{vanilla-flavouring}).

\subsubsection{What is Vanilla Oleoresin?}
It is a semi-solid concentrate obtained by removing the solvent from
the vanilla extract. A solution of isopropanol is frequently used
instead of ethanol for the maceration. Vanilla oleoresin has lost 
part of its aroma --hence its flavour-- during the removal of the
solvent.

\subsubsection{What is Vanilla Powder?}
\label{vanilla-powder}
Powdered vanilla beans. It might be pure, but normally it is
adulterated with vanilla oleoresin, sugar, food starch, or gum acacia.

\subsubsection{What is Vanilla-Vanillin Extract Flavouring and Powder?}
A combination of synthetic vanillin and vanilla oleoresin to create
extract and flavouring (see \ref{vanilla-extract},
\ref{vanilla-flavouring}, \ref{vanilla-powder}).

\subsubsection{What is Perfumery Vanilla Tincture?}
Similar to vanilla extract (see \ref{vanilla-extract}) but prepared
with perfumery alcohol, with near 90\% ethanol content. It is not
intended for consumption.

\subsubsection{What is Vanilla Absolute?}
It is the most concentrated form of vanilla. ``It is 7-13 times stronger
than good-quality vanilla beans but it has less well-rounded
character'' \cite{Purseglove-2}.

\subsection{Major types of Vanilla}

\subsubsection{What are vanilla splits?}
Whole bean that burst open during fermentation, and are frosted with
vanillin crystals \cite{Andy}.

\subsubsection{What are vanilla cuts?}
Beans that have been cut into pieces to accelerate the curing
process. This category might include small beans.

\subsubsection{What is Mexican Vanilla?}
%``Mexico has been regarded has the supplier of the vanilla with the
%finest aroma and flavour'' \cite{Purseglove-2} (please notice that it . 
It is supplied in 5 grades (or 7 if intermediate grades are included)
of whole beans and in the form of cuts. The top grades of Mexican
beans are rarely ``frosted'' with a surface coating of naturally
exuded vanillin.\cite{Purseglove-2}

\subsubsection{What is Bourbon vanilla?}
``It has a deeper `body' flavour than Mexican vanilla, but less fine
aroma'' \cite{Purseglove-2}. It is produced in Madagascar, the Comoro
Islands and R�union.

\subsubsection{What is Indonesian vanilla?}
The main source of Indonesian vanilla is Java. ``Java vanilla possesses
a deep, full-bodied flavour and is frequently used for blending with
synthetic vanillin'' \cite{Purseglove-2}

\subsubsection{What is South American or West Indian Vanilla?}
More similar in properties to Bourbon than to Mexican vanilla.

\subsubsection{What is Tahiti vanilla?}
It is obtained from {\em V. tahitensis\/} and ``possesses a
characteristic aromatic odour and usually has a lower vanillin content
than true vanilla.'' \cite{Purseglove-2}. It generally has less
flavour than true vanilla.

\subsubsection{What is Vanillons (Guadeloupe vanilla or Antilles vanilla)?}
It is obtained from {\em V. pompona\/}. ``Vanillons has a low vanillin
content and possesses a characteristic floral aroma, bearing
similarities to Tahiti vanilla'' \cite{Purseglove-2}. It has a poor
flavour and it is normally used in perfumery.



\subsubsection{Is it safe to buy Mexican vanilla?}


Mexican vanilla has one of the finest aromas, however, most of the
vanilla extract sold in Mexico is artificial. In M�xico there is almost a
complete lack of enforcement of labeling laws for vanilla.
Furthermore, nowhere in the world you can expect to buy a liter of
real vanilla extract for a couple of dollars. As a good example of
this kind of problems, I have seen turmeric being sold as saffron in a
well known supermarket. So don't be cheap: if you want good vanilla,
pay the price of getting it from a reliable source; if you care for
price, use artificial vanilla.

%In article <db0_9601230900@salata.com>,
%Joel Ehrlich <Joel.Ehrlich@salata.com> wrote:
%>
%>
%>The fact is that one source of artificial vanilla is a byproduct from
%>the conversion of woodpulp to paper. That is not, in and of itself,
%>dangerous. Some artificial vanilla is quite safe. Some is most decidedly
%>unsafe.
%>
%>What is dangerous is the complete lack of enforcement of any labelling
%>laws for vanilla in Mexico. You can call anything you want any thing you
%>wish. You can include anything in the bottle and label it or not as you
%>choose. And you can make any claims you wish on the bottle label.
%>
%>It isn't that there are no statutes regulating labelling in Mexico, it's
%>simply that there is no effort made to enforce them (other than on
%>bottled water).
%>
%>Now then, if were in Mexico and you held a bottle labelled "Pure
%>Vanilla" and you knew that there is very little real vanilla available
%>in Mexico and that which is available is very, very, very expensive,
%>would you buy that 1 liter bottle for less than \$5.00? Especially
%>knowing that there is no effort made to assure that what is on the label
%>is what is in the bottle?
%>
%>Joel


\subsection{For the do-it-yourselfer}

\subsubsection{How do I prepare Vanilla Extract?}

Juan San Mames shared the following recipe \cite{sanmames}:

\begin{quote}
  {\em
Use one vanilla bean for every 120 ml.  of any clear liquor (vodka 
preferably). With a knife, split the bean open  (always put your 
finger behind the knife). If the bean is hard, just break it into pieces. 
Then  put the bean in the liquor.  

Close the bottle and leave it for about two weeks or until the vanilla
bean aroma begins to come through. 

When you use the extract, if you don't want the vanilla seeds to show  
with the ingredients, use a coffee filter. You can return the 
seeds to the bottle.  If you make ice cream, you may want to show the 
seeds in the finished ice cream.}

\end{quote}

Bruce Steinberg added \cite{bruce}:

\begin{quote}
{\em
You can shake the bottle several times a week to accelerate the
extraction. Brandy may also be used for interesting variations.}
\end{quote}

According to US regulations, 1 l. of vanilla extract must contain a
minimum of 100 gr. of vanilla beans (I reckon that each
  regular size complete bean must weight between 3 and 5 gr.) of no
more or 25\% moisture content. Commercial extracts also include sugar
and glycerine, to help to ``fix'' the aroma \cite{Purseglove-2}.

\subsubsection{How do I prepare vanilla sugar?}
Store 1 or 2 vanilla beans on an air-tight jar of granulated
sugar. Allow one month for the flavour to permeate. If the beans are
always topped with sugar, the beans last for years. Use this sugar in
sweet dishes.\cite{Mulherin}

Storage temperature can be raised to 15-21 Celsius without detriment to
the flavour quality of the beans.\cite{Purseglove-2}

\subsubsection{How do I store my cured vanilla beans?}
Vanilla beans should be stored in open containers at a temperature of
about 10 C at a low humidity \cite{merory-56}

\subsubsection{How do I use vanilla in my kitchen?}
Use vanilla sugar to give a nice flavour to your drinks. It also
enhances the flavour of chocolate \cite{Mulherin}.

Almost any sweet dish will improve its flavour with a touch of vanilla
extract.

\subsection{Further information}
An excellent treatment of the topic can be found at
\cite{Purseglove-2}. 

The \emph{``Vanilla Cookbook''} by Patricia Rain is a complete book
from a more practical point of view. This book is out of print and 
a new edition is being written. You can contact her to get a notice
whenever the new edition of the book is available.

\begin{quote}
Patricia Rain's {\em "The Vanilla Cookbook"\/} covers the use of
vanilla from basic extracts through liqueurs, desserts,
souffles, and baking of all kinds, to full-tilt savoury recipes such
as "Seafood-Pecan Salad with Vanilla Mayonnaise", "Rice with Coconut,
Vanilla, Dates, and Lemon" and "Fresh Tuna Grenobleoise with
Vanilla". \cite{bruce}.
\end{quote}

The Patricia Rain's {\em Vanilla Information Hotline\/} is available
at
%408/457-0902
(408) 476-9111 --fax (408) 476-9112--
%(or fax at 408/457-2521) 
for any vanilla questions, to request a basic vanilla FAQ (by fax or
snail mail), or to get further info on ordering Tahitian and Bourbon
beans, extracts.
%, or "The Vanilla Cookbook" itself.

\section{Saffron}

\subsection{What is saffron?}
Saffron is the dried stigmas of the crocus sativus.  It is of orange
color and has a strong perfume and a bitter taste.  Saffron production
is regulated by ISO with standard 3632.

%Spain is one of the biggest growers of saffron in the world. Spanish
%saffron is divided into three categories: Mancha, Rio, Sierra-- Mancha
%being the best.  There is a better quality called Coupe (Short), but
%it is hard to find because of its limited production. Iran is also a
%big producer, but because of the embargo its saffron is not available
%in the US. Iranian #1 qualiti is usually all red, ant it looks like
%little arrow heads. Again, it is illegal to import it.  Basically, all
%saffrons are the same.  They all come from the Crocus Sativus, with
%slight variations in the countries where they are grown.  The
%selection of the saffron is what is most important.  The best Spanish
%saffron is the one that shows the ISO-3632 norm on the label,.  ISO is
%the International Standards Organization, based in Switzerland.  Spain
%is one of the subscribers to this organization.

\subsection{Why is saffron so expensive?}
Every plant has on average 3 flowers; each flower only 3 stigmas.  It
takes between 200,000 and 300,000 stigmas to make 1 kg. of saffron
\cite{Mulherin}. Pure saffron, however, has a strong flavour and a
pinch is sufficient for most dishes.

%For instance, one ounce of
%saffron will be sufficient to give aroma and flavour to 907/4 cups of
%dry rice.%\cite{saffron-faq}

Avoid buying powdered saffron, it might be adulterated.



%Antes de utilizar el azafr�n, envolvedlo en papel de aluminio y
%calentadlo durante cinco minutos en una sart�n, a fuego suave. A
%continuaci�n desle�dlo en un poco, (muy poco), de agua templada y
%dejadlo reposar diez minutillos antes de a�adirlo a lo que est�is
%cocinando.  Y si vuestro paladar es atrevido, probad a poner una hebra
%de azafr�n en el arroz con leche...  �Os sorprendera!.

\subsection{Why should I not use wooden utensils to work with
  saffron?}
Wood has an absorbing property. Since saffron is expensive you don't
want to waste it.

\subsection{What is Mexican saffron?}
Mexican saffron is the flower of {\em Carthamus tinctorius L. \/}
which is an annual herb grown in the temperate regions of Central
M�xico. Its quality is quite inferior to real saffron but it has
similar coloring properties. It is far cheaper.

\subsection{How do I store saffron?}
Saffron is sensitive to light and moisture. Keep it in a dark
container away from sunlight. It will last for years.

\subsection{Where is saffron native from?}

It is believed that it is native of Asia Minor.


\subsection{Further information}

%%Patricia Rain's {\em "The Vanilla Cookbook"\/} covers the use of
%%vanilla in everything from basic extracts through liqueurs, desserts,
%%souffles, and baking of all kinds, to full-tilt savoury recipes such
%%as "Seafood-Pecan Salad with Vanilla Mayonnaise," "Rice with Coconut,
%%Vanilla, Dates, and Lemon" and "Fresh Tuna Grenobleoise with
%%Vanilla" \cite{bruce}.

\emph{Vanilla, Saffron Imports} prints a pamphlet called \emph{Cooking
  and \& with Saffron}, they also have a WWW page (
\htmladdnormallink{http://www.saffron.com/} {http://www.saffron.com/}
) with some facts about saffron, including a photospectrometry report.
They can be reached at {\em ``Vanilla Saffron Imports, 949 Valencia
  Street, San Francisco, CA 94110''}.

Baby Saffron (\htmladdnormallink{http://www.babysaffron.com}
{http://www.babysaffron.com}) is a company in India that produces fine
saffron. They have a facts sheet in their home page that includes
medical and cooking uses; what to look for in powdered saffron and
uses of the herb. Very informative. 


\section{What is coriander/cilantro/Chinese parsley?}
Coriander is the common name for coriandrum sativum (fam.
umbelliferae). It is an annual plant similar to parsley. It has
erect, furrowed solid, branched stems. The alternate bright green
leaves are pinnate or bipinnate, the lower ones are broader leaflets
than the upper ones, which are finely divided.
Coriander seeds are cream to brown spheres of 1-1.5 mm. in
diameter. In the culinary argot, it is common to refer to the plant as
cilantro and to the seeds as coriander.


\subsection{Where does the name coriander comes from?}

"There is uncertainty about the [source of the ]generic name,
Coriandrum; it might be derived from the Greek word "koris" (= bug), a
reference perhaps to the plant's smell and the apperance of the
fruits.''  \cite{herbs}

%\subsection{Does cilantro taste to soap?}



\subsection{How do I store fresh cilantro?}
Different people have suggested different methods. Here is a list of
the most common ones.

\begin{itemize}
\item ``bouquet'' in the fridge. Cover loosely with plastic. 
\item ``bouquet'' in the window.
\item ``airtight container'' in the fridge. 
\item ``wrapping'' in damp towels, inside a plastic bag.
\end{itemize}

Sophie Laplante (sophie@cs.uchicago.edu) performed some experiments
on these different methods for storing cilantro. She found that the
airtight container seemed to keep it {\em edible\/}[sic] for the
longest time (3 weeks).


\section{Other Spices}

\subsection{Is there any substitute to coconut milk?}

You can probably find coconut milk in an Asian store, either in liquid
or powdered form. If you have no other choice, you can follow this
recipe \cite{ashall}:

\begin{quote}
  {\em
    \begin{itemize}
    \item Take a handful of shredded coconut and pack it in the bottom of
      a bowl.
    \item Pour boiling (and I mean really boiling) water just to cover the packed
      coconut and let stand until the water is cool.
    \item  Strain the coconut shreds
      and press them in the bottom of the strainer get as much liquid as possible.
    \end{itemize}
    The liquid is very close to coconut milk and will impart the flavor
    very well.
    }
\end{quote}

\section{Storing Spices}

3 factors affect the quality of stored spices:

\begin{itemize}
\item Light. Spices containing color pigments (such as capsicums, saffron,
  green cardamoms, turmeric) and chorophyll (dryed herbs) need
  protection from light. For instance, the color of capsicums is mostly
  due to carotenoids, which are photosensitive and oxidate in the
  presence of light.
\item Humidity. Since most spices are sold dry, they tend to attract water
  and mold.
\item Oxygen. The essential oil of spices oxydates in the presence of 
  atmospheric oxygen, specially at high temperatures. However, most
  whole spices are protected by a pericarp and their natural
  antioxidants which they contain. 
\end{itemize}

%Different spices have different storage requirements. 

Torricelli (cited in \cite{Pruthi}), studied
the loss of essential oil in the following spices: anise, cardamon,
coriander, fennel, cumin, sweet marjoram, mace, cloves, pepper,
allspice, and cinnamon. When the spices where kept in small paper bags
(containing 1 to 5g), in the dark for 5 years, they lost 47\% essential
oil on the average. In case of powder spices, they lost an average of
62\% and up to 90\%. The same spices, when kept, during six years in
dark glass containers, lost 24\% of their essential oils on the
average. When the containers were hermetic, and the spice filled the
container, the loss was from 0 to 5\%, whether the spice was powdered
or whole. 

So keep your spices in dark, sealed containers. Fill each container
completely. Put them in a fresh place (some people use the fridge, see~\ref{sec:storfridge}) and
away from light.  And your spices will long enough (whatever that
means to you).


\subsection{Should I store my spices in the fridge?}
\label{sec:storfridge}
Some people store spices inside the fridge. The
fridge keeps the spices in a dark, low temperature environment, hence
protecting them from light and rapid oxidation. There is only one
problem, whenever you open the spice container, humidity immediately
condenses on the surface of the spice and the container, then you
close the container and the moisture is kept captive. Humidity is a
natural enemy of most dry spices.

The fridge suits the bill if you do keep big containers there, from
which you regularly fill small-daily use ones. Nonetheless, for the
majority of the spices, it is more practical to buy small amounts of
each spice every once in a while, which in effect, guarantees their
freshness.

Some people freeze small sealed envelopes, each one storing a
``dose''. Therefore, they don't have the condensation problem.

\subsection{Bay leaves}

Bay leaves lose approximately 30\% of their volatile oil and 40-60\%
of their chlorophyll during one year of storage.  A good way to
evaluate the quality of the leaf is to determine how bright its color
is.

\subsection{Ground spices}
Ground spices, with greater surface exposed, tend to lose their
volatile oils. They also deteriorate faster than whole spices.

The needs for packaging vary  from spice to spice. In general, follow
the next guidelines:

\begin{itemize}
\item Use dark, air tight containers.
\item Fill the container as much as you can.
\item Avoid buying ground spices. Grind them yourself using a
  mortar.
\end{itemize}

\section{Others}

\subsection{Disclaimer}
This FAQ is provided as is without any express or implied warranties.
While every effort has been taken to ensure the accuracy of the
information contained in this article, the maintainer assumes no
responsibility for errors or omissions, or for damages resulting from
the use of the information contained herein.


\subsection{List of Contributors}

\begin{itemize}

\item Hall, Andrew S. (ashall@magnus.acs.ohio-state.edu), for his
  recipe to prepare coconut milk.
\item Laplante, Sophie (sophie@cs.uchicago.edu), for her
  research on cilantro storage.
\item Pforzheimer, Andy (apforz@pfood.win.net), for his
  corrections regarding vanilla extract and vanilla splits.
\item San Mames, Juan (VMPK89A@prodigy.com), for sharing his
  knowledge regarding vanilla.
\item Stafford, Maureen (stafford@csg.uwaterloo.ca), for
  proofreading the first draft (version 0.1) of this document.
\item Steinberg, Bruce (bruces\@sco.com), for his comments on how
  to prepare vanilla extract.
\end{itemize}

\bibliography{spices}

\end{document}
%Local IspellParsing: latex-mode
%LocalWords:  dmg,flavour,flavouring FAQ
%Local IspellDict: english

%Chinese five spices:gound star anise, fennel, cinnamon, cloves and Szechwan pepper

% Capsicums
% How do I tell whether a capsicum is hot? Taste it. Or stick to the
%% mild varieties

%The Whole Chile Pepper Book", Dewitt & Gerlach, Little Brown 1990
%lists:  Habanero (Capisicum chinense)  "In Spanish, the word Habanero
%means 'Havana-like,' or possibly 'from Havana,' . . . other . . .
%names . . . are 'Scot's Bonnet' or 'Scotch Bonnet,' (commonly used in
%the English-speaking Caribbean islands such as Jamaica) and 'Bahamian'
%or 'Bahama Mama' in the Bahamas."  So now we can add `Jamaquino
%Habanero'
%   ...a rose by any other name...  :-)

%I hadn't been familiar with these peppers, but I was lucky enough to
%be invited to cook at a local New Haven CT chinese resteraunt on
%occasion (a long story). The owner used to recieve shippments of these
%peppers still on the bush. We handled them with 'kid gloves'
%literally.
