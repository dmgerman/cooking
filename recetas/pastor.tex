\subsection{Tacos al Pastor}

Fuente: Salvador Sahag�n (Taqueria  aol.com)

\subsubsection{Receta del Adobo para los Tacos alpastor}

Ingredientes

\begin{itemize}
\item 10 chiles Pasilla
\item 10 chiles Guajillo
\item 1/2 ajo pelado
\item 1/4 de litro de vinagre Blanco destilado
\item 1/4 de cucharada de cominos
\item 5 clavos de olor
\item sal al gusto
\end{itemize}

Preparaci�n

\begin{enumerate}
\item Se limpian los chiles y se ponon a cocer en el vinagre hasta
  estar los chiles suaves y se muelen en el mismo vinagre junto con el
  ajo y los demas condimentos, hasta crear una pasta de buena
  consistensia ( casi igual a la del cemento para pegar ladrillo)
  aniada mas vinagre si es necesario. una vez que este la pasta bien
  molida proceda a cocerse nuevamente hasta que hierva meneandola
  constantemente para que no se pegue (����cuidado!!!! se pone muy
  caliente y brinca, si te cae una gota te chinga bien gacho). dejese
  que se enfrie . compre 1 kilo de carne de puerco (no lmporta el
  corte,solo que este bisteseada bien delgadita). unte la pasta en la
  carne cuidando de no poner en exeso (como si te pusieras bronceador)
  y coloque los bisteces uno encima del otro, tratando de figurar un
  trompo. Deje reposar la carne por lo menos 5 horas.
  
\item En un comal o cazuela caliente, coloque la carne, un bistec a la
  vez (si gusta puede poner un poco de aceite a la cazuela para que no
  se pegue) y se sancocha o sea se medio coce. Una vez cocida se corta
  en pedazos peque�os 1/4 de cent�metro aprox. y se vuelve a cocinar
  esta vez acompa�ado de una cebolla grande picada lo m�s fina
  posible. El olor inconfundible le dira que esta listo.
\end{enumerate}

Nota: si le gusta con Pinia y a falta de trompo agregue un poco de
jugo al adobo antes de untarlo en la carne.

Nota 2: como la mayoria de presonas no tienen trompo esta receta es lo
mas cercano a la receta original con trompo. esta es mi primera receta
que escribo asi que lo mas probable es que tendra fallas. si tienen
preguntas ya saben...no me molesten:-). tal vez seria mas facil si te
mandara el adobo ya preparado. espero y esto te sirva. que otra
quieres?

De Salvador Sahag�n (Taqueria aol.com)




%%% Local Variables: 
%%% mode: latex
%%% TeX-master: "recetas"
%%% End: 
