\section{Antojitos}


\subsection{Enchiladas De Aguascalientes}

Fuente: T�cnicas de la Alta Cocina, de Luisa Calder�n, Copyright 1979 \cite{tecnicas}

\begin{itemize}
\item tortillas medianas delgaditas
\item manteca o aceite
\item 10 chiles anchos medianos
\item 10 pimientas gorda
\item 1 raja canela
\item 6 clavos
\item 5 ajos medianos
\item 200 grs. queso rallado
\item sal 
\item pimienta
\item 2 cebollas desflemadas en rajas grandes
\item 10 chiles en vinagre 
\item 1/4 kilo cueritos en vinagre
\item 1 longaniza o chorizo
\item 3 calabacitas grandes
\item 3 papas blancas grandes
\item 4 zanahorias grandes
\item 1/4 kilo ejotes tiernos
\item 1 lechuga orejona
\end{itemize}

Manera de hacerse:

\begin{enumerate}
\item Se cuecen las verduras peladas y picadas  en cuadritos.  Aparte se
lavan los chiles, se  asan y limpian de venas y pepitas y se ponen a remojar
un agua hirviendo hasta que  se suavicen.  Se muelen con todas las especies.
Si se desea se puede agregar un huevo   entero crudo a esta salsa para que
se adhiera mejor.
\item Las tortillas (de la v�spera) se introducen en esta salsa y de ah�
se pasan a fre�r en  la manteca o aceite caliente donde se habr�     frito
antes el chorizo.  Se van pasando a un plat�n y rellen�ndose de cebolla y
queso  rallado.  Al final se espolvorean con la longaniza frita y las
verduras cocidas, los cueritos en vinagre y los chiles encurtidos.  Se  mete
al horno unos minutos y al salir se decora con la lechuga.
\end{enumerate}

\subsection{Guacamole}

Fuente: Diario Reforma del d�a Sept. 1, 1995.

\begin{itemize}
\item      2 aguacates
\item      2 tomates finamente picados
\item      1/2 cebolla finamente picada
\item      Sal y gotas de lim�n al gusto
\end{itemize}

Preparaci�n: Retire de la c�scara la pulpa de aguacate, ayud�ndose con un tenedor;
mach�quela.

En un plat�n coloque primero el aguacate, luego la cebolla y al �ltimo
el tomate; sazone con sal y lim�n y acompa�e con totopos.

\subsection{Enchiladas de queso}

Fuente: Diario Reforma del d�a Sept. 1, 1995.

\begin{itemize}
\item      5 tortillas para enchiladas
\item      100 gramos de queso panela o canasta rallado o desmoronado
\item      1 papa cocida
\item      1/2 taza de manteca de puerco
\item      1 chile jalape�o para torearlo
\item      Cebolla finamente picada al gusto
\end{itemize}

Preparaci�n: Ponga sobre fuego lento la manteca a que se caliente;
pase por �sta las tortillas, una por una, y luego rell�nelas con el
queso y forme los taquitos.

Aparte, pique la papa en cuadritos y fr�ala en la misma manteca junto
con el chile; s�rvalas sobre las enchiladas; espolvoree queso y
cebollita al gusto.

\subsection{Enchiladas tradicionales}

Fuente: Diario Reforma del d�a Sept. 1, 1995.

\begin{itemize}
\item      1 chile ancho dorado en aceite
\item      1 chile cascabel hervido en agua
\item      2 dientes de ajo
\item      1 huevo chico
\item      1/2 taza de leche
\item      1 cucharada de consom� de pollo
\item      1/2 kilo de tortillas blancas
\item      1/4 kilo de queso panela o canasta desmoronado
\item      1/2 taza de manteca de puerco
\end{itemize}

      Preparaci�n: Ponga en la licuadora los chiles, los ajos, el
huevo, la leche y el consom� de pollo; licue todo muy bien y vac�e la
salsa en un recipiente.

      Caliente la manteca en una sart�n; pase las tortillas una por
una por la salsa y luego por la manteca; rellene con el queso y
enrolle; acompa�e con chiles toreados y papas y zanahoria cocidas.

