
\subsection{Adobo}

Fuente: familia Lazo-Franco

\begin{itemize}
\item Chile ancho.
\item 1 diente de ajo.
\item Or�gano (lo que se tome con las yemas de tres dedos.)
\item Sal al gusto.
\end{itemize}

Cocer la carne que se haya escogido. En la licuadora se pone un poco del caldo que se obtuvo de la
cocci�n de la carne. El ajo y el or�gano se muelen con el chile ancho (que previamente se ha
desvenado, y puesto a remojar en un poco de agua con media cucharadita de sal -para quitarle lo
picoso.), agregando sal al gusto.

La consistencia de esta salsa debe ser no muy espesa, pero tampoco muy l�quida. 

Con la salsa se sirva la carne previamente cocida y al servi

\subsection{Pollo con Apio}

De la familia Lazo-Franco

\begin{itemize}
\item 1  Pollo de 1.5 a 2 kgs., limpio, partido en piezas.
\item 50 g  De Mantequilla o margarina
\item 1  Apio tierno, quitadas las hojas y las hebras y cortado en rebanaditas
   de 1 cm. aproximadamente.
\item 1/2   Cebolla grande picada finamente.
\item    Sal y pimienta al gusto
\end{itemize}

En una cacerola se calienta la mantequilla (sin dejar quemarse) y se dora el pollo. Una vez dorado,
se agrega la cebolla y el apio, moviendo el pollo para que no se queme. Cuando est� semidorado, se
agrega una taza de agua y sal al gusto. 

\subsection{Pastel de Manzana 5 Estrellas}

De la Familia Lazo-Kohlmann

Ingredientes


\begin{itemize}
\item 1        Barra de pan de caja (sin corteza)
\item 1        kg de manzana Golden (de la amarilla)
\item 1/2      barra de mantequilla (50 g)
\item 4        claras de huevo batidas a punto de turr�n
\item          Canela en polvo.
\end{itemize}
Elaboraci�n

Se calienta el horno a 250 grados cent�grados. Se engrasa con la mantequilla un molde de vidrio
refractario, rectangular. Se pone una capa de pan, a cubrir todo el fondo.

Las manzanas se mondan y se parten en rebanadas de aproximadamente 0.5 cm. de ancho Se
coloca una capa de manzanas.

Ahora se agrega una capa de clara de huevo batida. Se le espolvorea un poco de canela en polvo.

Se repite la acci�n una vez mas. A la capa final de clara de huevo se le agrega az�car revuelta con la
canela.

Se mete al horno por media hora, bajando la temperatura a 200 grados, s�lo para que la cubierta de
encima tome color dorado. Si se desea, se puede servir con helado vainilla.

\subsection{Sopa de Papa y tocino al estilo de mi suegra}

De la familia Lazo-Franco

Picar tocino y dorarlo. En una cacerola vaciar el exceso de grasa y
sofreir 1 + kg. de papas peladas y cortadas en cuadritos. Agregar 1
lt. de agua y una cucharada sopera de perejil picado finamente.
Agregar sal al gusto.

\begin{quote}
``Por lo que respecta a tu sopa de papa con tocino, por lo general se utilizan
como base para las sopas caldos de res o de pollo (verdaderos no en polvo),
y para quienes no les guste el caldo por la grasa, este se puede preparar
previamente y cuando la grasa se endurece y hace una especie de "costra" la
retiras, por aquello del colesterol y todas esas cosas.'' Ruth Serrano

\end{quote}


\subsection{Mole Verde}

De la Familia Manrique-Medell�n

Ingredients:

\begin{itemize}
\item 150 grs. de pepita verde
\item un pu�o grande de ajonjol� tostado
\item un pu�o (unos 100 grms) de almendra
\item pan duro (1/2 bolillo) frito
\item 1.5 tortillas frito
\item 6 pimientas
\item 6 clavos
\item pu�o chiquito de an�s (1 cucharada sopera m�s o menos)
\item 1 ajo grande
\item una rajita de canela de 3cm, una sola capa
\item 1 kg. de tomate verde crudo
\item 1 pu�o de chile verde
\item 1 pu�o cilantro (hojas, no tallos)
\end{itemize}

Se muele tomate, cilantro y chile en la licuadora. Aparte de muelen todas los dem�s ingredientes. En
una cazuela se agregan las especies, pepita, se sazona con un poco de caldo de pollo, se agrega el
tomate y se deja hervir. Agregue carne de pollo o de puerco al gusto. 

%%% Local Variables: 
%%% mode: latex
%%% TeX-master: "recetas"
%%% End: 
